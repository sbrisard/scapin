\documentclass[draft, appendixprefix=true, chapterprefix=true, fontsize=12pt, numbers=noendperiod]{scrbook}
\addtokomafont{disposition}{\rmfamily}
\addtokomafont{caption}{\footnotesize}
\setkomafont{descriptionlabel}{\normalfont\bfseries}
\setcapindent{0em}

\title{scapin}
\author{Sébastien Brisard}

\usepackage{amsfonts}
\usepackage[fleqn]{amsmath}
\setlength{\mathindent}{1ex}
\usepackage{amssymb}

\usepackage{amsthm}

\usepackage{csquotes}

\usepackage{polyglossia}
\setdefaultlanguage{english}

\usepackage[backend=biber,bibencoding=utf8,doi=false,giveninits=true,isbn=false,maxnames=10,minnames=5,sortcites=true,style=authoryear,texencoding=utf8,url=false]{biblatex}
\addbibresource{scapin.bib}

\usepackage{unicode-math}
\setmainfont{XITS}
\setmathfont{XITS Math}

\usepackage[breaklinks=true,colorlinks=true,linktocpage=true, pdfauthor={Sébastien Brisard}, pdfsubject={},pdftitle={Scapin},unicode=true,urlcolor=blue]{hyperref}

\usepackage[notref, notcite]{showkeys}
\renewcommand{\showkeyslabelformat}[1]{\color{red}\normalfont\scriptsize\ttfamily#1}

\usepackage{stmaryrd}

\newcommand{\cellindices}{\mathcal P}
\DeclareMathOperator{\conj}{conj}
\DeclareMathOperator{\curl}{\textbf{rot}}
\newcommand{\D}{{\mathrm d}}
\newcommand{\dbldot}{\mathbin{\mathord{:}}}
\DeclareMathOperator{\dft}{DFT}
\newcommand{\I}{\mathrm{i}}
\newcommand{\integers}{\mathbb{Z}}
\newcommand{\integrable}{\mathrm{L}^2}
\newcommand{\naturals}{\mathbb N}
\newcommand{\reals}{\mathbb{R}}
\newcommand{\symgrad}{\tens\epsilon}
\newcommand{\tr}{\operatorname{tr}}
\newcommand{\E}{\mathrm e}
\AtBeginDocument{ % See http://tex.stackexchange.com/questions/117990/
  \DeclareMathOperator{\asym}{\symbfsf{asym}}
  \let\div\undefined
  \DeclareMathOperator{\div}{div}
  \DeclareMathOperator{\vdiv}{\textbf{div}}
  \DeclareMathOperator{\tdiv}{\textbf{\textsf{div}}}
  \DeclareMathOperator{\grad}{\symbf{grad}}
  \DeclareMathOperator{\tgrad}{\symbfsf{grad}}
  \DeclareMathOperator{\sym}{\symbfsf{sym}}
  \newcommand{\tens}[1]{\symbfsf{#1}}
  \renewcommand{\vec}[1]{\symbf{#1}}
  \newcommand{\vecup}[1]{\symbf{#1}}

  \newcommand{\PI}{\symup{\pi}}
}
\DeclareMathOperator{\sinc}{sinc}
\newcommand{\tensors}{\mathcal T}
\newcommand{\stresses}{\mathcal S}
\newcommand{\strains}{\mathcal E}
\newcommand{\todo}[1]{\color{red}TODO~---~#1\color{black}}
\newcommand{\tuple}[1]{\mathsf{#1}}

\begin{document}
\maketitle

\chapter{Nomenclature}
\label{cha:20204224074257}

\begin{itemize}
\item \(d\): dimension of the physical space (typically \(d=2, 3\))
\item \(\Omega\): \(d\)-dimensional unit-cell
\item \(L_1,\ldots, L_d\): dimensions of the unit-cell:
  \(\Omega=(0, L_1)\times(0, L_2)\times\cdots\times(0, L_d)\)
\item \(\lvert\Omega\rvert=L_1L_2\cdots L_d\): volume of the unit-cell
\item \(\tuple{n}\): \(d\)-dimensional tuple of integers
  \(\tuple{n}=(n_1, n_2, \ldots, n_d)\)
\item \(\tuple{N}=(N_1, N_2, \ldots, N_d)\): size of the simulation grid
\item \(\lvert N\rvert=N_1N_2\cdots N_d\): total number of cells
\item \(h_i=L_i/N_i\): size of the cells (\(i=1, \ldots, d\))
\item
  \(\cellindices=\{0, \ldots, N_1-1\}\times\{0, \ldots, N_2-1\}\times\{0,
  \ldots, N_3-1\}\): set of cell indices
\item \(\Omega_{\tuple{p}}\): cells of the simulation grid
  (\(\tuple{p}\in\cellindices\))
\end{itemize}

\chapter{Continuous Green operators}
\label{chap:202002060528}

In this chapter, we discuss various boundary-value problems in a periodic
setting. For each of these problems, we introduce the associated \emph{Green
  operator}.

\section{On Fourier series}

Owing to the periodic setting, the fields that are involved in the various BVPs
to be discussed in this chapter are expanded in Fourier series. \(\tens T\)
being a \(\Omega\)-periodic tensor field
\begin{equation}
  \tens T(\vec x)=\sum_{\tuple{n}\in\integers^d}\mathcal F(\tens T)(\vec
  k_{\tuple{n}})\exp(\I\vec k_{\tuple{n}}\cdot\vec x),
\end{equation}
where \(\tuple{n}\) denotes a \(d\)-dimensional tuple of integers (see
chapter~\ref{cha:20204224074257}). The wave vectors \(\vec k_{\tuple{n}}\) are
given by
\begin{equation}
  \label{eq:3}
  \vec k_{\tuple{n}}=\frac{2\pi n_1}{L_1}\vec e_1+\frac{2\pi n_2}{L_2}\vec e_2+
  \cdots+\frac{2\pi n_d}{L_d}\vec e_d,
\end{equation}
and the Fourier coefficients of \(\tens T\) are defined as follows
\begin{equation}
  \label{eq:5}
  \mathcal F(\tens T)(\vec k)=\frac1V\int_{\vec x\in\Omega}\tens T(\vec x)
  \exp(-\I\vec k\cdot\vec x)\,\D x_1\cdots\D x_d.
\end{equation}

It is recalled that the Fourier coefficients of the gradient and divergence of
\(\tens T\) can readily be computed from the Fourier coefficients of
\(\tens T\)
\begin{equation}
  \label{eq:4}
  \mathcal F(\tgrad\tens T)(\vec k)=\mathcal F(\tens T)(\vec k)\otimes\vec k
  \quad\text{and}\quad
  \mathcal F(\tdiv\tens T)(\vec k)=\mathcal F(\tens T)(\vec k)\cdot\vec k.
\end{equation}

When no confusion is possible, we will use the tilde to denote the Fourier
coefficients: \(\tilde{\tens T}_n=\mathcal F(\tens T)(\vec k_n)\).

\section{Conductivity}

\section{Elasticity}

We first define a few functional spaces; \(\tensors_2(\Omega)\) denotes the
space of second-order, symmetric, tensor fields, with square-integrable
components. Then, the space \(\tens\stresses(\Omega)\) of periodic,
self-equilibrated stresses is defined as follows
\begin{equation}
  \label{eq:7}
  \tens\sigma\in\stresses(\Omega)\iff\left\{
  \begin{gathered}
    \tens\sigma\in\tensors_2(\Omega)\\
    \div\tens\sigma=\vec 0\text{ a.e in }\Omega\\
    \tens\sigma\cdot\vec e_i\text{ is }L_i\vec e_i\text{-periodic for all }i=1, 2, \ldots, d\text{ (no summation),}
  \end{gathered}
  \right.
\end{equation}
where the last condition expresses the periodicity of tractions in all
directions parallel to the sides of the unit-cell. The space
\(\tens\strains(\Omega)\) of periodic, geometrically compatible strains is
defined as follows
\begin{equation}
  \label{eq:8}
  \tens\varepsilon\in\strains(\Omega)\iff\left\{
  \begin{gathered}
    \tens\varepsilon\in\tensors_2(\Omega)\\
    \tens\varepsilon=\sym\grad\vec u\text{ a.e. in }\Omega\text{ for some vector
      field }\vec u\\
    \vec u\text{ has square-integrable components}\\
    \vec u\text{ is }\Omega\text{-periodic.}
  \end{gathered}
  \right.
\end{equation}
Finally, we define the spaces of stresses and strains with zero average
\begin{equation}
  \label{eq:9}
  \stresses_0(\Omega)=\bigl\{\tens\sigma\in\stresses(\Omega),
  \langle\tens\sigma\rangle=\tens0\bigr\}
  \quad\text{and}\quad
  \strains_0(\Omega)=\{\tens\varepsilon\in\strains(\Omega),
  \langle\tens\varepsilon\rangle=\tens0\bigr\}.
\end{equation}


We are now ready to define the periodic, fourth-order Green operator for
strains \(\tens\Gamma\). Let \(\tens C\) be the homogeneous elastic stiffness
of the body \(\Omega\)\footnote{In other words, \(\tens C\) is a constant,
  fourth-order tensor with major and minor symmetries; furthermore, \(\tens C\)
  is positive definite.}. Let \(\tens\tau\in\tensors_2(\Omega)\) be a
prescribed tensor field (\emph{stress-polarization}). We want to find the
equilibrium state of the body \(\Omega\), subjected to the eigenstress
\(\tens\tau\) and periodic boundary conditions. In other words, we want to find
the solution to the following problem
\begin{equation}
  \label{eq:6}
  \text{Find }\tens\sigma\in\stresses_0(\Omega)
  \text{ and }\tens\varepsilon\in\strains_0(\Omega)
  \text{ such that }\tens\sigma=\tens C\dbldot\tens\varepsilon+\tens\tau
  \text{ a.e. in }\Omega.
\end{equation}

Owing to the periodic boundary conditions, we use Fourier series expansions of
\(\tens\tau\), \(\tens\sigma\), \(\tens\varepsilon\) and \(\vec u\)
\begin{equation}
  \label{eq:10}
  \begin{Bmatrix}
    \tens\tau(\vec x)\\
    \tens\sigma(\vec x)\\
    \tens\varepsilon(\vec x)\\
    \vec u(\vec x)
  \end{Bmatrix}
  =\sum_{n\in\integers^d}
  \begin{Bmatrix}
    \tilde{\tens\tau}_n\\
    \tilde{\tens\sigma}_n\\
    \tilde{\tens\varepsilon}_n\\
    \tilde{\vec u}_n
  \end{Bmatrix}
  \exp(\I \vec k_n\cdot\vec x).
\end{equation}

The Fourier modes \(\tilde{\tens\sigma}_n\), \(\tilde{\tens\varepsilon}_n\) and
\(\tilde{\vec u}_n\) solve the following equations (respectively: equilibrium,
geometric compatibility, constitutive relation)
\begin{subequations}
  \label{eq:11}
  \begin{gather}
    \label{eq:15}
    \tilde{\tens\sigma}_n\cdot\vec k_n=\vec 0\\
    \label{eq:13}
    \tilde{\tens\varepsilon}_n=\frac{\I}{2}\bigl(\tilde{\vec u}_n\otimes\vec k_n
    +\vec k_n\otimes\tilde{\vec u}_n\bigr)\\
    \label{eq:12}
    \tilde{\tens\sigma}_n=\tens C\dbldot\tilde{\tens\varepsilon}_n
    +\tilde{\tens\tau}_n.
  \end{gather}
\end{subequations}

Plugging Eq.~\eqref{eq:12} into Eq.~\eqref{eq:13}, and recalling that
\(\tens C\) has the minor symmetries, we find the following expression of
\(\tilde{\tens\sigma}\)
\begin{equation}
  \label{eq:14}
  \tilde{\tens\sigma}_n=\I\bigl(\tens C\cdot\vec k_n\bigr)\cdot\tilde{\vec u}_n
  +\tilde{\tens\tau}_n.
\end{equation}

The Cauchy stress tensor being symmetric, Eq.~\eqref{eq:15} also reads
\(\vec k\cdot\tilde{\tens{\sigma}}=\vec 0\) and
\begin{equation}
  \label{eq:16}
  \tilde{\vec u}_n=\I\bigl(\vec k_n\cdot\tens C\cdot\vec k_n\bigr)^{-1}
  \cdot\tilde{\tens\tau}_n\cdot\vec k_n
\end{equation}
which delivers the following expression for the Fourier modes of the strain
field
\begin{equation}
  \label{eq:17}
  \tilde{\tens\varepsilon}_n=-\tfrac12\bigl[\bigl(\vec k_n\cdot\tens C\cdot
  \vec k_n\bigr)^{-1}\cdot\tilde{\tens\tau}_n\cdot\vec k_n\bigr]\otimes\vec k_n
  -\tfrac12\vec k_n\otimes\bigl[\bigl(\vec k_n\cdot\tens C\cdot\vec k_n
  \bigr)^{-1}\cdot\tilde{\tens\tau}_n\cdot\vec k_n\bigr].
\end{equation}

The above relation defines a linear mapping between \(\tilde{\tens\tau}_n\) and
\(\tilde{\tens\varepsilon}_n\). For each Fourier mode \(n\), we therefore
introduce the fourth-order tensor \(\tilde{\tens\Gamma}_n\) with major and
minor symmetries, such that
\(\tilde{\tens\varepsilon}_n=-\tilde{\tens\Gamma}_n\dbldot{\tilde{\tens\tau}}_n\). From
Eq.~\eqref{eq:18}, it results that
\(\tilde{\tens\Gamma}_n=\hat{\tens\Gamma(\vec k)}\) where, for arbitrary
wave-vector \(\vec k\), \(\hat{\tens\Gamma}(\vec k)\) is a fourth-order tensor
with major and minor symmetries, such that
\begin{equation}
  \label{eq:18}
  \hat{\tens\Gamma}(\vec k)\dbldot\tilde{\tens\tau}=\tfrac12\bigl[\bigl(\vec n
  \cdot\tens C\cdot\vec n\bigr)^{-1}\cdot\tilde{\tens\tau}\cdot\vec n\bigr]
  \otimes\vec n+\tfrac12\vec n\otimes\bigl[\bigl(\vec n\cdot\tens C\cdot\vec n
  \bigr)^{-1}\cdot\tilde{\tens\tau}\cdot\vec n\bigr],
\end{equation}
where \(\vec n=\vec k/\lVert\vec k\rVert\). Eq.~\eqref{eq:18} defines
\(\hat{\tens\Gamma}(\vec k)\) by how it operates on second-order, symmetric
tensors. A closed-form expression of this tensor can be derived in the case of
an isotropic material, for which
\begin{equation}
  \label{eq:19}
  \tens C=\lambda\tens I_2\otimes\tens I_2+2\mu\tens I_4,
\end{equation}
where \(\tens I_2\) (resp. \(\tens I_4\)) is the second-order
(resp. fourth-order) identity tensor, and \(\lambda\), \(\mu\) are the Lamé
coefficients. Then
\begin{equation}
  \label{eq:20}
  \vec n\cdot\bigl(\tens I_2\otimes\tens I_2\bigr)\vec n=\vec n\otimes\vec n
\end{equation}
then (recalling that \(\lVert\vec n\rVert=1\))
\begin{equation}
  \label{eq:21}
  \begin{aligned}[b]
    \vec n\cdot\tens I_4\cdot\vec n&=\tfrac12 n_i\bigl(\delta_{ik}\delta_{jl}+
    \delta_{il}\delta_{jk}\bigr)n_l\vec e_j\otimes\vec e_k=\tfrac12\bigl(n_kn_j
    +n_in_i\delta_{jk}\bigr)\vec e_j\otimes\vec e_k\\
    &=\tfrac12\bigl[\vec n\otimes\vec n+\bigl(\vec n\cdot\vec n\bigr)\tens I_2
    \bigr]=\tfrac12\bigl(\vec n\otimes\vec n+\tens I_2\bigr)
    =\vec n\otimes\vec n+\tfrac12\bigl(\tens I_2-\vec n\otimes\vec n\bigr)
  \end{aligned}
\end{equation}
and finally
\begin{equation}
  \label{eq:22}
  \vec n\cdot\tens C\cdot\vec n=\bigl(\lambda+2\mu\bigr)\vec n\otimes\vec n+\mu
  \bigl(\tens I_2-\vec n\otimes\vec n\bigr)=2\mu\frac{1-\nu}{1-2\nu}\vec n
  \otimes\vec n+\mu\bigl(\tens I_2-\vec n\otimes\vec n\bigr),
\end{equation}
where \(\nu\) denotes the Poisson ratio. The above second-order tensor is
easily inverted, since \(\vec n\otimes\vec n\) and
\(\tens I_2-\vec n\otimes\vec n\) are two orthogonal projectors (in the sense
of the ``\(\dbldot\)'' product)
\begin{equation}
  \label{eq:23}
  2\mu\bigl(\vec n\cdot\tens C\cdot\vec n\bigr)^{-1}=\frac{1-2\nu}{1-\nu}\vec n
  \otimes\vec n+2\bigl(\tens I_2-\vec n\otimes\vec n\bigr)=2\tens I_2-\frac1
  {1-\nu}\vec n\otimes\vec n,
\end{equation}
from which it results that
\begin{equation}
  \label{eq:24}
  2\mu\bigl(\vec n\cdot\tens C\cdot\vec n\bigr)^{-1}\cdot\tilde{\tens\tau}\cdot
  \vec n=2\tilde{\tens\tau}\cdot\vec n-\frac{\vec n\cdot\tilde{\tens\tau}\cdot
    \vec n}{1-\nu}\vec n
\end{equation}
and finally
\begin{equation}
  \label{eq:25}
  2\mu\hat{\tens\Gamma}(\vec k)\dbldot\tilde{\tens\tau}=\bigl(\tilde{\tens\tau}
  \cdot\vec n\bigr)\otimes\vec n+\vec n\otimes\bigl(\tilde{\tens\tau}\cdot\vec n
  \bigr)-\frac{\vec n\cdot\tilde{\tens\tau}\cdot\vec n}{1-\nu}\vec n\otimes\vec
  n.
\end{equation}

The components of the \(\hat{\tens\Gamma}\) tensor are then readily found
\begin{equation}
  \label{eq:26}
  \hat{\Gamma}_{ijkl}(\vec k)=\frac1{4\mu}\bigl(\delta_{ik}n_jn_l+\delta_{il}n_j
  n_k+\delta_{jk}n_in_l+\delta_{jl}n_in_k\bigr)-\frac{n_in_jn_kn_l}{2\mu\bigl(1-
    \nu\bigr)},
\end{equation}
which coincide with classical expressions \parencite[see
e.g.][]{suqu1990}. Implementation of Eq.~\eqref{eq:26} is cumbersome; it is
only used for testing purposes. For the implementation of the Green operator
itself, we use Eq.~\eqref{eq:25}.

\section{Hyperelasticity}

\chapter{Discrete Green operators}

In this chapter, we introduce various discretizations of the Green operator~;
we will adopt the vocabulary of linear elasticity, although the concepts apply
to all the types of physics presented in Chapter~\ref{chap:202002060528}.

\section{On the discrete Fourier transform}

Let \(x=(x_{\tuple{p}})\) be a finite set of scalar values indexed by the
\(d\)-tuple \(\tuple{p}=(p_1, \ldots, p_d)\) where \(0\leq p_i<N_i\) (\(N_i\)
is the number of data points in the \(i\)-th direction). The discrete Fourier
transform is a discrete set of scalar values \(\dft_{\tuple{n}}(x)\)
indexed by the \(d\)-tuple \(\tuple{n}\in\integers^d\), defined as follows
\begin{equation}
  \dft_{\tuple{n}}(x)=\sum_{p_1=0}^{N_1-1}\cdots\sum_{p_d=0}^{N_d-1}\exp
  \Bigl[-2\I\PI\Bigl(\frac{n_1p_1}{N_1}+\cdots+\frac{n_dp_d}{N_d}\Bigr)\Bigr]
  x_{\tuple{p}}.
\end{equation}

Note that in the above definition, no restrictions are applied to the
multi-index \(\tuple{n}\). However, it can be verified that the above series of
tensors is in fact \(\tuple{N}\)-periodic:
\(\dft_{\tuple{n}+\tuple{N}}(x)=\dft_{\tuple{n}}(x)\), where
\(\tuple{n}+\tuple{N}=(n_1+N_1, \ldots, n_d+N_d)\). Therefore, the
\(\tuple{n}\)-index is effectively restricted to \(0\leq n_i<N_i\) as well. The
most important results concerning the DFT are the \emph{inversion formula}
\begin{equation}
  x_{\tuple{p}}=\frac1{\lvert\tuple{N}\rvert}\sum_{n_1=0}^{N_1-1}\cdots
  \sum_{n_d=0}^{N_d-1}\exp\Bigl[2\I\PI\Bigl(\frac{n_1p_1}{N_1}+\cdots+
  \frac{n_dp_d}{N_d}\Bigr)\Bigr]\dft_{\tuple{n}}(x)
\end{equation}
and the circular convolution theorem, which states that
\begin{equation}
  \dft_{\tuple{n}}(x\ast y)=\dft_{\tuple{n}}(x)\dft_{\tuple{n}}(y),
\end{equation}
where \(z=x\ast y\) denotes the convolution product
\begin{equation}
  (x\ast y)_{\tuple p}=\sum_{q_1=0}^{N_1-1}\cdots\sum_{q_d=0}^{N_d-1}
  x_{\tuple{q}}y_{\tuple{p}-\tuple{q}}.
\end{equation}

The DFT is readily extended to tensor data points. In the absence of ambiguity,
the shorthand \(\hat{x}_{\tuple{n}}\) will be adopted for
\(\dft_{\tuple{n}}(x)\).

To close this section, we observe that the DFT of a series of \emph{real} data
points is a series of \emph{complex} data points. However, these complex values
have the following property
\begin{equation}
  \dft_{\tuple{N}-\tuple{n}}(x)=\conj\dft_{\tuple{n}}(x).
\end{equation}
The above condition is actually a \emph{necessary and sufficient} condition for
the \(x_{\tuple{p}}\) to be real.

\section{The approximation space}

In order to define a discrete Green operator, we need to introduce the
approximation space for the stress-polarizations. We will consider here
stress-polarizations that are constant over each cell of a regular grid of size
\(N_1\times\cdots\times N_d\). The cells of this grid are
\begin{equation}
  \Omega_{\tuple{p}}^{\tuple{h}}=\{p_ih_i\leq x_i<\bigl(p_i+1\bigr)h_i, i=1,
  \ldots, d\},
  \qquad\text{\emph{(no summation on \(i\)).}}
\end{equation}
where \(\tuple{p}=(p_1,\ldots,p_d)\in\cellindices\) denotes a \(d\)-tuple of
integers, such that \(0\leq p_i<N_i\), \(i=1,\ldots, d\) and \(h_i=L_i/N_i\) is
the cell-size in the \(i\)-th direction. The total number of cells is
\(\lvert N\rvert=N_1\cdots N_d\).

We consider discrete stress-polarizations \(\tens\tau^{\tuple{h}}\) that are
constant over each cell of the grid: \(\tens\tau_{\tuple{p}}^{\tuple{h}}\)
denotes the constant value of \(\tens\tau^{\tuple{h}}\) in cell
\(\Omega_{\tuple{p}}^{\tuple{h}}\). The \(\tuple{h}\) superscript reminds that
\(\tens\tau^{\tuple{h}}\) is a discrete approximation of the true
stress-polarization \(\tens\tau\). We will call this approximation subspace:
\(\tensors_2^{\tuple{h}}(\Omega)\).

As discussed \todo{xref}, the discrete Green operator is defined as the
restriction to this approximation space of the continuous Green operator, seen
as a bilinear form, or an approximation of it. In other words, we want to
propose an approximation of the quantity
\begin{equation}
  \label{eq:20203427053434}
  \langle\tens\varpi^{\tuple{h}}\dbldot\tens\Gamma(\tens\tau^{\tuple{h}})\rangle
  \simeq\langle\tens\varpi^{\tuple{h}}\dbldot\tens\Gamma^{\tuple{h}}(\tens
  \tau^{\tuple{h}})\rangle\quad\text{for all }\tens\tau^{\tuple{h}},
  \tens\varpi^{\tuple{h}}\in\tensors_2^{\tuple{h}}(\Omega),
\end{equation}
where \(\Gamma^{\tuple{h}}\) is defined only over
\(\tensors_2(\Omega)\). \(\Gamma^{\tuple{h}}\) can therefore be seen as a
linear mapping between the cell values \(\tens\tau_{\tuple{p}}^{\tuple{h}}\) of
\(\tens\tau^{\tuple{p}}\) and the cell values of
\(\tens\Gamma^{\tuple{h}}(\tens\tau^{\tuple{h}})\); \(\Gamma^{\tuple{h}}\) is
therefore a \emph{matrix}, and Eq.~\eqref{eq:20203427053434} should be
understood as
\begin{equation}
  \label{eq:20204027054030}
  \langle\tens\varpi^{\tuple{h}}\dbldot\tens\Gamma(\tens\tau^{\tuple{h}})\rangle
  \simeq\frac1{\lvert\tuple{N}\rvert}\sum_{\tuple{p}, \tuple{q}\in\cellindices}
  \tens\varpi_{\tuple{p}}^{\tuple{h}}\dbldot
  \tens\Gamma_{\tuple{p}\tuple{q}}^{\tuple{h}}\dbldot\tens
  \tau_{\tuple{q}}^{\tuple{h}}\quad\text{for all }\tens\tau^{\tuple{h}},
  \tens\varpi^{\tuple{h}}\in\tensors_2^{\tuple{h}}(\Omega).
\end{equation}

The continuous Green operator is translation invariant, and this property will
of course be transferred to the ``exact'' discrete Green operator~; we will in
fact require \emph{all} dicretizations of the Green operator to have this
property. In other words,
\(\tens\Gamma_{\tuple{p}\tuple{q}}^{\tuple{h}}=\tens\Gamma_{\tuple p-\tuple q}^{\tuple{h}}\)
and Eq.~\eqref{eq:20204027054030} reads
\begin{equation}
  \langle\tens\varpi^{\tuple{h}}\dbldot\tens\Gamma(\tens\tau^{\tuple{h}})\rangle
  \simeq\frac1{\lvert\tuple{N}\rvert}\sum_{\tuple{p}, \tuple{q}\in\cellindices}
  \tens\varpi_{\tuple{p}}^{\tuple{h}}\dbldot
  \tens\Gamma_{\tuple p-\tuple q}^{\tuple{h}}\dbldot\tens
  \tau_{\tuple{q}}^{\tuple{h}}\quad\text{for all }\tens\tau^{\tuple{h}},\tens
  \varpi^{\tuple{h}}\in\tensors_2^{\tuple{h}}(\Omega).
\end{equation}

Note that \(\tens\Gamma_{\tuple{p}}^{\tuple{h}}\) is now indexed by \emph{one}
index only and its DFT can be introduced unambiguously

\begin{equation}
  \label{eq:20201527081546}
  \begin{aligned}[b]
    &\frac{1}{\lvert\tuple N\rvert}\sum_{\tuple{p}, \tuple{q}\in\cellindices}
    \tens\varpi_{\tuple{p}}^{\tuple{h}}\dbldot\tens\Gamma_{\tuple p-\tuple q}^{\tuple h}
    \dbldot\tens\tau_{\tuple{q}}^{\tuple{h}}\\
    ={}&\frac1{\lvert\tuple N\rvert^2}\sum_{\tuple p, \tuple q, \tuple n\in\cellindices}
    \exp\Bigl[2\I\PI\sum_{j=1}^d\frac{n_j}{N_j}\bigl(p_j-q_j\bigr)\Bigr]
    \tens\varpi_{\tuple{p}}^{\tuple{h}}\dbldot\hat{\tens\Gamma}_{\tuple n}^{\tuple h}
    \dbldot\tens\tau_{\tuple q}^{\tuple h}\\
    ={}&\frac{1}{\lvert\tuple N\rvert^2}
    \sum_{\tuple n\in\cellindices}\Bigl\{\Bigl[\sum_{\tuple{p}\in\cellindices}
    \exp\Bigl(2\I\PI\sum_{j=1}^d\frac{n_jp_j}{N_j}\Bigr)
    \tens\varpi_{\tuple p}^{\tuple h}\Bigr]\dbldot\hat{\tens\Gamma}_{\tuple n}^{\tuple h}
    \dbldot\Bigl[\sum_{\tuple q\in\cellindices}\exp\Bigl(-2\I\PI\sum_{j=1}^d
    \frac{n_jq_j}{N_j}\tens\tau_{\tuple q}^{\tuple{h}}\Bigr)\Bigr]\Bigr\}.
  \end{aligned}
\end{equation}

Since \(\tens\varpi\) is real, we have
\(\conj\tens\varpi_{\tuple p}^h=\tens\varpi_{\tuple p}^h\) and the first sum in
square brackets in Eq.~\eqref{eq:20201527081546} reads
\begin{equation}
  \sum_{\tuple{p}\in\cellindices}\exp\Bigl(2\I\PI\sum_{j=1}^d\frac{n_jp_j}{N_j}\Bigr)
  \tens\varpi_{\tuple p}^{\tuple h}=\conj\sum_{\tuple{p}\in\cellindices}\exp
  \Bigl(-2\I\PI\sum_{j=1}^d\frac{n_jp_j}{N_j}\Bigr)\tens\varpi_{\tuple p}^{\tuple h}
  =\conj\hat{\tens\varpi}_{\tuple n}^{\tuple h},
\end{equation}
while the second sum in square brackets reduces to
\(\hat{\tens\tau}_{\tuple n}^{\tuple h}\). Gathering the above results, we find
\begin{equation}
  \frac{1}{\lvert\tuple N\rvert}\sum_{\tuple p, \tuple q\in\cellindices}
  \tens\varpi_{\tuple p}^{\tuple h}\dbldot\tens\Gamma_{\tuple p-\tuple q}^{\tuple h}
  \dbldot\tens\tau_{\tuple q}^{\tuple h}=\frac1{\lvert\tuple N\rvert^2}
  \sum_{\tuple n\in\cellindices}\conj\hat{\tens\varpi}_{\tuple n}^{\tuple h}
  \dbldot\hat{\tens\Gamma}_{\tuple n}^{\tuple h}\dbldot
  \hat{\tens\tau}_{\tuple n}^{\tuple h}.
\end{equation}

The above equation can be understood as follows. \(\tens\Gamma^{\tuple h}\) is
a linear operator that maps the cell-wise constant field
\(\tens\tau^{\tuple h}\) to the cell-wise constant field
\(\tens\eta^{\tuple h}\), the cell-values of which are given by their DFT
\begin{equation}
  \tens{\eta}_{\tuple p}^{\tuple h}=
  \dft^{-1}_{\tuple p}(\hat{\tens\eta}_{\bullet}^{\tuple h}),\quad\text{with}
  \quad\hat{\tens\eta}_{\tuple n}^{\tuple h}
  =\hat{\tens\Gamma}_{\tuple n}^{\tuple h}\dbldot
  \hat{\tens\tau}_{\tuple n}^{\tuple h}.
\end{equation}

Then, from the Plancherel theorem \todo{xref}
\begin{equation}
  \frac1{\lvert\tuple N\rvert^2}
  \sum_{\tuple n\in\cellindices}\conj\hat{\tens\varpi}_{\tuple n}^{\tuple h}
  \dbldot\hat{\tens\Gamma}_{\tuple n}^{\tuple h}\dbldot
  \hat{\tens\tau}_{\tuple n}^{\tuple h}=\frac1{\lvert\tuple N\rvert^2}
  \sum_{\tuple n\in\cellindices}\conj\hat{\tens\varpi}_{\tuple n}^{\tuple h}
  \dbldot\hat{\tens\eta}_{\tuple n}^{\tuple h}=\frac1{\lvert\tuple N\rvert}
  \sum_{\tuple p\in\cellindices}\tens\varpi_{\tuple p}^{\tuple h}
  \dbldot\tens\eta_{\tuple n}^{\tuple h}
\end{equation}
and the last sum can be seen as the volume average
\(\langle\tens\varpi^{\tuple h}\dbldot\tens\eta^{\tuple
  h}\rangle\). Remembering that this expression was proposed as an
approximation of
\(\langle\tens\varpi^{\tuple h}\dbldot\tens\Gamma(\tens\tau^{\tuple h})\rangle\),
we finally find
\begin{equation}
  \langle\tens\varpi^{\tuple h}\dbldot\tens\Gamma(\tens\tau^{\tuple h})\rangle\simeq\langle\tens\varpi^{\tuple h}\dbldot\tens\eta^{\tuple h}\rangle=\langle\tens\varpi^{\tuple h}\dbldot\tens\Gamma^{\tuple h}(\tens\tau^{\tuple h})\rangle\quad\text{for all }\tens\varpi^{\tuple h}\in\tensors_2^{\tuple h}(\Omega),
\end{equation}
from which we find
\begin{equation}
  \tens\Gamma(\tens\tau^{\tuple h})\simeq\tens\Gamma^{\tuple h}
  (\tens\tau^{\tuple h}).
\end{equation}

The discrete Green operator, which was first introduced as an approximation of
the continuous Green operator, seen as a bilinear form, can also be understood
as an approximation of the continuous Green operator, seen as a linear
mapping. This latter point of view will become extremely efficient when it
comes to discretizing the Lippmann--Schwinger equation.

It results from the above developments that an explicit expression of the
discrete Green operator as a (gigantic) matrix is never needed. Instead, the
matrix-vector product
\(\tens\tau^{\tuple h}\mapsto\tens\Gamma^{\tuple h}(\tens\tau^{\tuple h})\) is
implemented in a matrix-free fashion as follows
\begin{enumerate}
\item\label{item:20202818112815} Given
  \(\tens\tau^{\tuple h}\in\tensors_2^{\tuple h}(\Omega)\), compute the
  discrete Fourier transform \(\hat{\tens\tau}_{\tuple n}^{\tuple h}\) of its
  cell-values:
  \(\hat{\tens\tau}_{\tuple n}^{\tuple h}=\dft_{\tuple
    n}(\tens\tau_\bullet^{\tuple h})\),
\item for each discrete frequency, compute
  \(\hat{\tens\eta}_{\tuple n}^{\tuple h}=\hat{\tens\Gamma}_{\tuple n}^{\tuple
    h}\dbldot\hat{\tens\tau}_{\tuple n}^{\tuple h}\),
\item\label{item:20202818112832} compute the inverse discrete Fourier transform
  \(\tens\eta_{\tuple{p}}^{\tuple h}\) of \(\hat{\tens\eta}_{\tuple{n}}^{\tuple h}\),
\end{enumerate}
discrete Fourier transforms being computed in steps~\ref{item:20202818112815}
and \ref{item:20202818112832} by means of the FFT.

\bigskip

In the remainder of this chapter, we propose various discretizations of the
Green operator. Before we proceed, though, it should be emphasized that the
discrete Green operator must map a \emph{real} field onto a \emph{real}
field. In other words, we must have
\(\hat{\tens\eta}_{\tuple N-\tuple n}^{\tuple h}=\conj\hat{\tens\eta}_{\tuple
  n}^{\tuple h}\) for all \(\tuple n\). Since
\(\hat{\tens\eta}_{\tuple n}^{\tuple h} =\hat{\tens\Gamma}_{\tuple n}^{\tuple
  h}\dbldot \hat{\tens\tau}_{\tuple n}^{\tuple h}\) and
\(\hat{\tens\tau}_{\tuple n}\) already satisfies this condition (it is the DFT
of a \emph{real} field), any discrete operator that will be considered below
must ensure that
\begin{equation}
  \hat{\tens\Gamma}_{\tuple N-\tuple n}^{\tuple h}=\conj\hat{\tens\Gamma}_{\tuple
    n}^{\tuple h}\quad\text{for all }\tuple n\in\cellindices.
\end{equation}

\todo{start from here}

In order to solve the Lippmann--Schwinger equation, we
need to evaluate \(\tens\Gamma(\tens\tau^h)\). Since \(\tens\tau^h\) is
cell-wise constant, its Fourier coefficients can be computed explicitly, and
exact evaluation of \(\tens\Gamma(\tens\tau^h)\) can be performed formally
\parencite{bris2010a,bris2012a} as follows
\begin{equation}
  \label{eq:20202118112132}
  \langle\tens\varpi^h\dbldot\tens\Gamma(\tens\tau^h)\rangle=\frac1{N^2}
  \sum_{n_1=0}^{N_1-1}\cdots\sum_{n_d=0}^{N_d-1}\conj(\hat{\tens\varpi}_
  {\tuple{n}}^h)\dbldot\hat{\tens\Gamma}_n^{h, \mathrm{c}}\dbldot
  \hat{\tens\tau}_{\tuple{n}}^h,
\end{equation}
where \(\hat{\tens\tau}_{\tuple{n}}^h\) denote the discrete Fourier transform
of the \(\tens\tau_{\tuple{p}}^h\)
\begin{equation}
  \hat{\tens\tau}_{\tuple{n}}^h=\sum_{p_1=0}^{N_1-1}\cdots\sum_{p_d=0}^{N_d-1}\exp\Bigl[
  -2i\pi\Bigl(\frac{p_1n_1}{N_1}+\cdots\frac{p_dn_d}{N_d}\Bigr)\Bigr]
  \hat{\tens\tau}_{\tuple{p}}^h
\end{equation}
and \(\hat{\tens\Gamma}_{\tuple{n}}^{h, \mathrm{c}}\) is the \emph{consistent}
discretization of the Green operator
\begin{equation}
  \label{eq:20202218112204}
  \hat{\tens\Gamma}_{\tuple{n}}^{h, \mathrm{c}}=\sum_{m\in\integers^d}\bigl[F(\vec
  \alpha_{\tuple{n+mN}})\bigr]^2\hat{\tens\Gamma}(\vec k_{\tuple{n+mN}}),
\end{equation}
where \(\tuple{n+mN}\) denotes the following \(d\)-tuple
\begin{equation}
  \tuple{n+mN}=\bigl(n_1+m_1N_1, \ldots, n_d+m_dN_d\bigr),
\end{equation}
while \(\alpha_{\tuple{n}}\) is the following dimensionless vector
\begin{equation}
  \vec\alpha_{\tuple{n}}=\frac{2\pi h_1n_1}{L_1}\vec e_1+\cdots+\frac{2\pi h_dn_d}{L_d}\vec e_d.
\end{equation}
Finally, \(F\) is the tensor product of sine cardinal functions
\begin{equation}
  F(\vec\alpha)=\sinc\frac{\alpha_1}2\cdots\sinc\frac{\alpha_d}2.
\end{equation}

Eq.~\eqref{eq:20202118112132} is not practical, as the infinite
series~\eqref{eq:20202218112204} can in general not be evaluated. However, it
shows that \(\tens\Gamma(\tens\tau^h)\) could in principle be computed very
efficiently in Fourier space as follows

It is important to note that, regardless of the exact definition of the discrete Green operator \(\hat{\tens\Gamma}_{\tuple{n}}^h\),  \(\tens\eta_{\tuple{p}}^h\) must be \emph{real}.

\section{The \texttt{fftfreq} function}

For \(n, N\in\naturals\), \(0\leq n<N\), we introduce \(Z(n, N)\)
\begin{equation}
  Z(n, N)=
  \begin{cases}
    n & \text{if }2n<N,\\
    n-N & \text{otherwise.}
  \end{cases}
\end{equation}
For \(n<0\) or \(n\geq N\), \(Z(n, N)\) is defined by
\(N\)-periodicity. \(Z\) is very similar to the
\texttt{numpy.fft.fftfreq} function. Then
\begin{equation}
  Z(N-n, N)=
  \begin{cases}
    Z(n) & \text{if }2n=N,\\
    -Z(n) & \text{otherwise.}
  \end{cases}
\end{equation}
Indeed
\begin{enumerate}
\item If \(n=0\), then \(Z(N-n, N)=Z(N, N)=Z(0, N)=0=-Z(n, N)\).
\item If \(N\) is even, \(N=2M\)
  \begin{enumerate}
  \item If \(0<n<M\)
    \begin{gather*}
      2n<N\quad\Rightarrow\quad Z(n, N)=n,\\
      M<N-n\quad\Rightarrow\quad N<2\bigl(N-n\bigr)\quad\Rightarrow\quad Z(N-n, N)=N-n-N=-n.
    \end{gather*}
  \item If \(n=M\)
    \begin{gather*}
      2n=N\quad\Rightarrow\quad Z(n, N)=n-N=-M,\\
      2(N-n)=2M=N\quad\Rightarrow\quad Z(N-n, N)=-M.
    \end{gather*}
  \item If \(M<n<N\)
    \begin{gather*}
      N<2n\quad\Rightarrow\quad Z(n, N)=n-N,\\
     N-n<M\quad\Rightarrow\quad 2\bigl(N-n\bigr)<N\quad\Rightarrow\quad Z(N-n, N)=N-n.
    \end{gather*}
  \end{enumerate}
\item If \(N\) is odd, \(N=2M+1\)
  \begin{enumerate}
  \item If \(0<n\leq M\)
    \begin{gather*}
      2n\leq 2M<N\quad\Rightarrow\quad Z(n, N)=n,\\
      M+1\leq N-n\quad\Rightarrow\quad N<2\bigl(N-n\bigr)\quad\Rightarrow\quad Z(N-n, N)=N-n-N=-n.
    \end{gather*}
  \item If \(M+1\leq n<N\)
    \begin{gather*}
      N+1\leq 2n\quad\Rightarrow\quad Z(n, N)=n-N\\
      N-n\leq N-M-1=M\quad\Rightarrow\quad 2\bigl(N-n\bigr)\leq 2M<N\quad\Rightarrow\quad Z(N-n, N)=N-n
    \end{gather*}
  \end{enumerate}
\end{enumerate}


\section{The truncated operator of \textcite{moul1998}}

\begin{equation}
  \hat{\tens\Gamma}_{\tuple{n}}^{h, \mathrm{MS98}}=\hat{\tens\Gamma}(\vec k_{\tuple{m}}),\quad\text{with}\quad
  m_i=Z(n_i, N_i),\quad i=1,\ldots, d.
\end{equation}


\printbibliography

\end{document}

%%% Local Variables:
%%% coding: utf-8
%%% fill-column: 79
%%% mode: latex
%%% TeX-engine: xetex
%%% TeX-master: t
%%% End:
