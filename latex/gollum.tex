\documentclass[draft, appendixprefix=true, chapterprefix=true, fontsize=12pt, numbers=noendperiod]{scrbook}
\addtokomafont{disposition}{\rmfamily}
\addtokomafont{caption}{\footnotesize}
\setkomafont{descriptionlabel}{\normalfont\bfseries}
\setcapindent{0em}

\title{Gollum}
\author{Sébastien Brisard}

\usepackage{amsfonts}
\usepackage[fleqn]{amsmath}
\setlength{\mathindent}{1ex}
\usepackage{amssymb}

\usepackage{amsthm}

\usepackage{csquotes}

\usepackage{polyglossia}
\setdefaultlanguage{english}

\usepackage[backend=biber,bibencoding=utf8,doi=false,giveninits=true,isbn=false,maxnames=10,minnames=5,sortcites=true,style=alphabetic-verb,texencoding=utf8,url=false]{biblatex}
%\addbibresource{.bib}

\usepackage{unicode-math}
\setmainfont{XITS}
\setmathfont{XITS Math}

\usepackage[breaklinks=true,colorlinks=true,linktocpage=true, pdfauthor={Sébastien Brisard}, pdfsubject={},pdftitle={Gollum},unicode=true,urlcolor=blue]{hyperref}

\usepackage[notref, notcite]{showkeys}
\renewcommand{\showkeyslabelformat}[1]{\color{red}\normalfont\scriptsize\ttfamily#1}

\usepackage{stmaryrd}

\DeclareMathOperator{\curl}{\textbf{rot}}
\newcommand{\D}{{\mathrm d}}
\newcommand{\dbldot}{\mathbin{\mathord{:}}}
\newcommand{\I}{\mathrm{i}}
\newcommand{\integers}{\mathbb{Z}}
\newcommand{\reals}{\mathbb{R}}
\newcommand{\symgrad}{\tens\epsilon}
\newcommand{\tr}{\operatorname{tr}}
\newcommand{\E}{\mathrm e}
\AtBeginDocument{ % See http://tex.stackexchange.com/questions/117990/
  \DeclareMathOperator{\asym}{\symbfsf{asym}}
  \let\div\undefined
  \DeclareMathOperator{\div}{div}
  \DeclareMathOperator{\vdiv}{\textbf{div}}
  \DeclareMathOperator{\tdiv}{\textbf{\textsf{div}}}
  \DeclareMathOperator{\grad}{\symbf{grad}}
  \DeclareMathOperator{\tgrad}{\symbfsf{grad}}
  \DeclareMathOperator{\sym}{\symbfsf{sym}}
  \newcommand{\tens}[1]{\symbfsf{#1}}
  \renewcommand{\vec}[1]{\symbf{#1}}
  \newcommand{\vecup}[1]{\symbf{#1}}

  \newcommand{\PI}{\symup{\pi}}
}

\begin{document}
\maketitle

\chapter{Theory}

In this chapter, we discuss various boundary-value problems in a periodic
setting. The \(d\)-dimensional periodic cell \(\Omega\subset\reals^d\) has
dimensions \(L_1\times L_2\times\cdots L_d\)
\begin{equation}
  \label{eq:1}
  \Omega=(0, L_1)\times(0, L_2)\times\cdots\times(0, L_d)
\end{equation}
and \(V=L_1L_2\cdots L_d\) denotes its (\(d\)-dimensional) volume.

Owing to the periodic setting, the fields that are involved in these various
BVPs will be expanded in Fourier series. \(\tens T\) being a
\(\Omega\)-periodic tensor field
\begin{equation}
  \label{eq:2}
  \tens T(\vec x)=\sum_{n\in\integers^d}\mathcal F(\tens T)(\vec k_n)\exp(\I
  \vec k_n\cdot\vec x),
\end{equation}
where the wave vectors \(\vec k_n\) are given by
\begin{equation}
  \label{eq:3}
  \vec k_n=\sum_{i=1}^d\frac{2\pi n_i}{L_i}\vec e_i,
\end{equation}
while the Fourier coefficients are defined as follows
\begin{equation}
  \label{eq:5}
  \mathcal F(\tens T)(\vec k)=\frac1V\int_{\vec x\in\Omega}\tens T(\vec x)
  \exp(-\I\vec k\cdot\vec x)\,\D x_1\cdots\D x_d.
\end{equation}

It is recalled that the Fourier coefficients of the gradient and divergence of
\(\tens T\) can readily be computed from the Fourier coefficients of
\(\tens T\)
\begin{equation}
  \label{eq:4}
  \mathcal F(\tgrad\tens T)(\vec k)=\mathcal F(\tens T)(\vec k)\otimes\vec k
  \quad\text{and}\quad
  \mathcal F(\tdiv\tens T)(\vec k)=\mathcal F(\tens T)(\vec k)\cdot\vec k.
\end{equation}

When no confusion is possible, we will use the tilde to denote the Fourier
coefficients: \(\tilde{\tens T}_n=\mathcal F(\tens T)(\vec k_n)\).

\section{Continuous Green operators}

\subsection{Conductivity}

\subsection{Elasticity}

\subsection{Hyperelasticity}

\end{document}

%%% Local Variables:
%%% coding: utf-8
%%% fill-column: 79
%%% mode: latex
%%% TeX-engine: xetex
%%% TeX-master: t
%%% End:
