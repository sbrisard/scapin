\documentclass[draft, appendixprefix=true, chapterprefix=true, fontsize=12pt, numbers=noendperiod]{scrbook}
\addtokomafont{disposition}{\rmfamily}
\addtokomafont{caption}{\footnotesize}
\setkomafont{descriptionlabel}{\normalfont\bfseries}
\setcapindent{0em}

\title{Gollum}
\author{Sébastien Brisard}

\usepackage{amsfonts}
\usepackage[fleqn]{amsmath}
\setlength{\mathindent}{1ex}
\usepackage{amssymb}

\usepackage{amsthm}

\usepackage{csquotes}

\usepackage{polyglossia}
\setdefaultlanguage{english}

\usepackage[backend=biber,bibencoding=utf8,doi=false,giveninits=true,isbn=false,maxnames=10,minnames=5,sortcites=true,style=alphabetic-verb,texencoding=utf8,url=false]{biblatex}
%\addbibresource{.bib}

\usepackage{unicode-math}
\setmainfont{XITS}
\setmathfont{XITS Math}

\usepackage[breaklinks=true,colorlinks=true,linktocpage=true, pdfauthor={Sébastien Brisard}, pdfsubject={},pdftitle={Gollum},unicode=true,urlcolor=blue]{hyperref}

\usepackage[notref, notcite]{showkeys}
\renewcommand{\showkeyslabelformat}[1]{\color{red}\normalfont\scriptsize\ttfamily#1}

\usepackage{stmaryrd}

\DeclareMathOperator{\curl}{\textbf{rot}}
\newcommand{\D}{{\mathrm d}}
\newcommand{\dbldot}{\mathbin{\mathord{:}}}
\newcommand{\I}{\mathrm{i}}
\newcommand{\integers}{\mathbb{Z}}
\newcommand{\integrable}{\mathrm{L}^2}
\newcommand{\reals}{\mathbb{R}}
\newcommand{\symgrad}{\tens\epsilon}
\newcommand{\tr}{\operatorname{tr}}
\newcommand{\E}{\mathrm e}
\AtBeginDocument{ % See http://tex.stackexchange.com/questions/117990/
  \DeclareMathOperator{\asym}{\symbfsf{asym}}
  \let\div\undefined
  \DeclareMathOperator{\div}{div}
  \DeclareMathOperator{\vdiv}{\textbf{div}}
  \DeclareMathOperator{\tdiv}{\textbf{\textsf{div}}}
  \DeclareMathOperator{\grad}{\symbf{grad}}
  \DeclareMathOperator{\tgrad}{\symbfsf{grad}}
  \DeclareMathOperator{\sym}{\symbfsf{sym}}
  \newcommand{\tens}[1]{\symbfsf{#1}}
  \renewcommand{\vec}[1]{\symbf{#1}}
  \newcommand{\vecup}[1]{\symbf{#1}}

  \newcommand{\PI}{\symup{\pi}}
}
\newcommand{\tensors}{\mathcal T}
\newcommand{\stresses}{\mathcal S}
\newcommand{\strains}{\mathcal E}

\begin{document}
\maketitle

\chapter{Theory}

In this chapter, we discuss various boundary-value problems in a periodic
setting. The \(d\)-dimensional periodic cell \(\Omega\subset\reals^d\) has
dimensions \(L_1\times L_2\times\cdots L_d\)
\begin{equation}
  \label{eq:1}
  \Omega=(0, L_1)\times(0, L_2)\times\cdots\times(0, L_d)
\end{equation}
and \(V=L_1L_2\cdots L_d\) denotes its (\(d\)-dimensional) volume.

Owing to the periodic setting, the fields that are involved in these various
BVPs will be expanded in Fourier series. \(\tens T\) being a
\(\Omega\)-periodic tensor field
\begin{equation}
  \label{eq:2}
  \tens T(\vec x)=\sum_{n\in\integers^d}\mathcal F(\tens T)(\vec k_n)\exp(\I
  \vec k_n\cdot\vec x),
\end{equation}
where the wave vectors \(\vec k_n\) are given by
\begin{equation}
  \label{eq:3}
  \vec k_n=\sum_{i=1}^d\frac{2\pi n_i}{L_i}\vec e_i,
\end{equation}
while the Fourier coefficients are defined as follows
\begin{equation}
  \label{eq:5}
  \mathcal F(\tens T)(\vec k)=\frac1V\int_{\vec x\in\Omega}\tens T(\vec x)
  \exp(-\I\vec k\cdot\vec x)\,\D x_1\cdots\D x_d.
\end{equation}

It is recalled that the Fourier coefficients of the gradient and divergence of
\(\tens T\) can readily be computed from the Fourier coefficients of
\(\tens T\)
\begin{equation}
  \label{eq:4}
  \mathcal F(\tgrad\tens T)(\vec k)=\mathcal F(\tens T)(\vec k)\otimes\vec k
  \quad\text{and}\quad
  \mathcal F(\tdiv\tens T)(\vec k)=\mathcal F(\tens T)(\vec k)\cdot\vec k.
\end{equation}

When no confusion is possible, we will use the tilde to denote the Fourier
coefficients: \(\tilde{\tens T}_n=\mathcal F(\tens T)(\vec k_n)\).

\section{Continuous Green operators}

\subsection{Conductivity}

\subsection{Elasticity}

We first define a few functional spaces; \(\tensors_2(\Omega)\) denotes the
space of second-order, symmetric, tensor fields, with square-integrable
components. Then, the space \(\tens\stresses(\Omega)\) of periodic,
self-equilibrated stresses is defined as follows
\begin{equation}
  \label{eq:7}
  \tens\sigma\in\stresses(\Omega)\iff\left\{
  \begin{gathered}
    \tens\sigma\in\tensors_2(\Omega)\\
    \div\tens\sigma=\vec 0\text{ a.e in }\Omega\\
    \tens\sigma\cdot\vec e_i\text{ is }L_i\vec e_i\text{-periodic for all }i=1, 2, \ldots, d\text{ (no summation),}
  \end{gathered}
  \right.
\end{equation}
where the last condition expresses the periodicity of tractions in all
directions parallel to the sides of the unit-cell. The space
\(\tens\strains(\Omega)\) of periodic, geometrically compatible strains is
defined as follows
\begin{equation}
  \label{eq:8}
  \tens\varepsilon\in\strains(\Omega)\iff\left\{
  \begin{gathered}
    \tens\varepsilon\in\tensors_2(\Omega)\\
    \tens\varepsilon=\sym\grad\vec u\text{ a.e. in }\Omega\text{ for some vector
      field }\vec u\\
    \vec u\text{ has square-integrable components}\\
    \vec u\text{ is }\Omega\text{-periodic.}
  \end{gathered}
  \right.
\end{equation}
Finally, we define the spaces of stresses and strains with zero average
\begin{equation}
  \label{eq:9}
  \stresses_0(\Omega)=\bigl\{\tens\sigma\in\stresses(\Omega),
  \langle\tens\sigma\rangle=\tens0\bigr\}
  \quad\text{and}\quad
  \strains_0(\Omega)=\{\tens\varepsilon\in\strains(\Omega),
  \langle\tens\varepsilon\rangle=\tens0\bigr\}.
\end{equation}


We are now ready to define the periodic, fourth-order Green operator for
strains \(\tens\Gamma\). Let \(\tens C\) be the homogeneous elastic stiffness
of the body \(\Omega\)\footnote{In other words, \(\tens C\) is a constant,
  fourth-order tensor with major and minor symmetries; furthermore, \(\tens C\)
  is positive definite.}. Let \(\tens\tau\in\tensors_2(\Omega)\) be a
prescribed tensor field (\emph{stress-polarization}). We want to find the
equilibrium state of the body \(\Omega\), subjected to the eigenstress
\(\tens\tau\) and periodic boundary conditions. In other words, we want to find
the solution to the following problem
\begin{equation}
  \label{eq:6}
  \text{Find }\tens\sigma\in\stresses_0(\Omega)
  \text{ and }\tens\varepsilon\in\strains_0(\Omega)
  \text{ such that }\tens\sigma=\tens C\dbldot\tens\varepsilon
  \text{ a.e. in }\Omega.
\end{equation}

\subsection{Hyperelasticity}

\end{document}

%%% Local Variables:
%%% coding: utf-8
%%% fill-column: 79
%%% mode: latex
%%% TeX-engine: xetex
%%% TeX-master: t
%%% End:
